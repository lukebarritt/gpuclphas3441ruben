\documentclass[twocolumn,a4paper]{article}
\usepackage{mathtools}
\usepackage[square]{natbib}
\usepackage{graphicx}
\usepackage[unicode,pagebackref,colorlinks]{hyperref}
\usepackage{SIunits}

\begin{document}
\title{Verification of Nuclear Disarmament, Challenges and Prospects}
\author{J.~Aldridge, L.~Barritt, V.~Christodoulou, S.~Lu, \\
	R.~Moors, J.~Rajroop, N.~Theodorou, K.~Xiao, \\
\small	Department of Physics and Astronomy, \\
\small	University College London, \\
\small	London WC1E 6BT, \\
\small	United Kingdom}
\date{10\textsuperscript{th} March 2012}
\maketitle

\onecolumn

\section*{Executive summary}
\subsection*{The Context}
Why do we Verify?  We verify to assure confidence. We have a
declaration, our verification methods, and our confidence. If we are
too intrusive with our verification methods there is the threat of the
spread of proliferation information which could in turn lead to a
breakdown of the declaration agreements alongside an increased risk of
mass devastation. If the verification methods employed are too feeble,
the uncertainty in the confidence will be high and inspectors may fail
to detect anomalous behaviour in the host’s disarmament activities.

This balance is known as the information barrier and it is spawn from
the consideration of ratified treaties, the capabilities of the
equipment and the feasibility of the procedure. The treaties outline
what information should and shouldn't be shared with the inspectors;
the equipment and procedure are the technological or economic
constraints.

*A chart of time, treaty and declaration information or disarmament
rates, there should be an inflection at 2002 *

Summer's treaty review
Nick's information barrier in the context of treaty
Luke's weapon introduction
Jack's confidence in verification

\subsection*{The Process}
What do we verify?
Luke's weapon introduction
Nick's brief overview 
Ralph's dismantling process
Valentino's chain of custody / Containment and surveillance of dismantlement 
Luke's Blending down
\subsection*{The Methods}
How do we verify?
Jenelle's Passive detection; Gamma signature, Neutron signature
Kaijian's detection methods; Induced Gamma and Neutron signature 
Nick's Pit stuffing
Luke's blending down
\subsection*{The Technology}
What do we use to verify and how does it work?
Valentino's Containment and surveillance / Ralph's review of tags and seals tech
Kaijian's detection methods
Jenelle's Passive detection
Valentino's Muon Tomography 

\subsection*{The Conclusion}
(What are the Strengths weaknesses opportunities and threats?)
\subsection*{Final Summary}

\newpage
\tableofcontents
\twocolumn

\section{Introduction}

\subsection{Facts on Nuclear Weapons}
Nuclear weapons are devices that generate a massive amount of energy
from nuclear reactions, used to create mass destruction. Their energy
is derived from two methods of bomb: nuclear fission or a combination
of fission and fusion. Weapons where the fission reaction generates
wholly the explosive output are called atomic bombs. The majority of
their energy occurs in fission reactions where a piece of sub-critical
material is `shot' into a mass of fissile material with a
supercritical mass. This begins an exponentially increasing chain
reaction, releasing energy. Another method uses compression of the
sub-critical sphere of fuel using an explosive to implode the
sphere. The two most commonly used fissile materials for atomic bombs
are (enriched) uranium-235 and plutonium-239. Uranium is said to be
enriched when the percentage of uranium-235 is increased by the
process of isotope separation.

The other type of nuclear weapon -- utilising the fusion reaction, is
the thermonuclear bomb (also known as hydrogen bombs).  The fusion
reaction typically occurs with deuterium and tritium to release
energy, but this is first triggered by a fission reaction to heat the
fusion fuel.  There are six countries purportedly that have conducted
thermonuclear weapons: the United States, Russia, United Kingdom, The
People's Republic of China, France and India.

Nuclear weapons were essential to maintaining international security
during the Cold War because they were a means of deterrence. The end
of the Cold War made the doctrine of mutual Soviet-American deterrence
obsolete.  Deterrence continues to be a relevant consideration for
many states with regard to threats from other states. But reliance on
nuclear weapons for this purpose is becoming increasingly hazardous
and decreasingly effective, such that nuclear weapons have no real
legitimate purpose in today’s world.  They are immoral to use, would
cause genocide, and their use is illegal because civilian casualties
would be inevitable. These weapons kill everything in their path and
also cause additional death through radioactive fallout. The effects
of detonating a weapon can spread for hundreds of miles, causing
long-term health problems for people not killed in the explosion.

In the 21st century around \$40 billion a year, or 10\% of the annual
US military budget, is spent on nuclear weapons. The US spent \$5.8
trillion on nuclear weapons between the early 1940s and 1996. Trident,
the UK’s nuclear weapons system, costs up to \$4 billion a year to
run, and plans to replace it will cost \$154 billion.

The likelihood that non-state terrorists will get their hands on
nuclear weaponry is increasing, increasing the need for
disarmament. In today's war waged on world order by terrorists,
nuclear weapons are the ultimate means of mass devastation. And
non-state terrorist groups with nuclear weapons are conceptually
outside the bounds of a deterrent strategy and present difficult new
security challenges.

There is also an economic incentive to dismantle nuclear weapon
stockpiles -- it has been estimated that worldwide costs exceeded \$1
trillion dollars for 2010-2011 [ref] This is in comparison to the
estimate that the cost of implementing a dismantlement program for
processing plutonium and highly enriched uranium into non-weapons
grade material would cost just \$7 billion per year for ten
years. This minimal amount in comparison is equivalent to less than
half of unaccountable spending in the Pentagon over the last decade,
0.09\% of current world military spending, or 25\% of the \$28 billion
spent every year to keep nuclear weapons secure.

Disassembly can last from a few days to a period of a few weeks,
depending on the bomb or type of warhead. [ref]. Ideally once the
dismantlement has taken place, the individual parts would be disposed
of or incinerated, so that they could not be reused. The difficulty in
the dismantlement process comes from verification of whether the
weapon is genuine without compromising sensitive national security
information.

Inspectors would not be allowed to know details of the size, shape and
composition etc. of the warhead. The quantity of fissile material in a
nuclear warhead is typically classified, so techniques have to be used
by the inspectors to ensure nothing is hidden when they are not able
to measure it in detail themselves.

One certain of the dismantlement process is the problem of dealing
with the hundreds of tons of plutonium and thousands of tons of highly
enriched uranium that the warheads contain. A way to deal with this
would be to build a specialised nuclear power reactor that could use
the plutonium and uranium as fuel. This process would generate
electric power and convert the fuel into a form that cannot be used in
nuclear weapons again.[ref]

\subsubsection{Nuclear Weapons States}
Although exact numbers are not known, there are around 20\,000 to
40\,000 nuclear weapons in the world [ref] These weapons have the
capability to destroy entire cities, murdering hundreds of thousands
of people in the process. These devices have no place in a peaceful
world and bring limitations to human development.

\begin{table}
  \begin{tabular}{|c|c|c|c|}
    \hline
    Country	& Active& Total		& CTBT	\\
    \hline
    USA		& 1950 	& 8500		& S	\\
    Russia	& 2430 	& 11000		& R	\\
    UK		& 160 	& 225		& R	\\
    France	& 290	& 300		& R	\\
    China	& 180	& 240		& S	\\
    India	&	& 90		& N	\\
    Pakistan	&	& 100		& N	\\
    North Korea	&	& \(<\)10	& N	\\
    Israel	&	& 140		& S	\\
    \hline
  \end{tabular}
  \caption{Number of nuclear warheads, active and total, by
    country. Status of ratification of the Comprehensive Test Ban
    Treaty (CTBT) is also shown. S denotes Signatory, R for
    Ratified and N for Non-Signatory.}
  \label{tab:NWSwarheadsCTBT}
\end{table}

\subsubsection{International Atomic Energy Agency (IAEA)}
The International Atomic Energy Agency (IAEA) is an international
organisation that seeks to promote the peaceful use of nuclear energy,
and to inhibit its use for any military purpose, including nuclear
weapons. The IAEA was established as an autonomous organisation on 29
July 1957. Though established independently of the United
Nations through its own international treaty, the IAEA Statute, the
IAEA reports to both the UN General Assembly and Security Council.

The IAEA serves as an intergovernmental forum for scientific and
technical cooperation in the peaceful use of nuclear
technology and nuclear power worldwide.  The programs of the IAEA
encourage the development of the peaceful applications of nuclear
technology, provide international safeguards against misuse of nuclear
technology and nuclear materials, and promote nuclear
safety (including radiation protection) and nuclear security standards
and their implementation.

The IAEA's mission is guided by the interests and needs of Member
States, strategic plans and the vision embodied in the IAEA
Statute. Three main pillars -- or areas of work -- underpin the IAEA's
mission: Safety and Security; Science and Technology; and Safeguards
and Verification

The IAEA executes this mission with three main functions: the
inspection of existing nuclear facilities to ensure their peaceful
use, providing information and developing standards to ensure the
safety and security of nuclear facilities, and as a hub for the
various fields of science involved in the peaceful applications of
nuclear technology.

The IAEA is generally described as having three main missions:
Peaceful uses: Promoting the peaceful uses of nuclear energy by its
member states, Safeguards: Implementing safeguards to verify that
nuclear energy is not used for military purposes, and Nuclear safety:
Promoting high standards for nuclear safety

\subsubsection{Non-Proliferation Treaty (NPT)}
The NPT is a landmark international treaty whose objective is to
prevent the spread of nuclear weapons and weapons technology, to
promote cooperation in the peaceful uses of nuclear energy and to
further the goal of achieving nuclear disarmament and general and
complete disarmament. The Treaty represents the only binding
commitment in a multilateral treaty to the goal of disarmament by the
nuclear-weapon States. Opened for signature in 1968, the Treaty
entered into force in 1970. On 11 May 1995, the Treaty was extended
indefinitely. A total of 190 parties have joined the Treaty, including
the five nuclear-weapon States. More countries have ratified the NPT
than any other arms limitation and disarmament agreement, a testament
to the Treaty's significance. Recalling the determination expressed by
the Parties to the 1963 Treaty banning nuclear weapons tests in the
atmosphere, in outer space and under water in its Preamble, its goal
is to achieve the discontinuance of all test explosions of nuclear
weapons for all time and to continue negotiations to this end.

Other goals are to further the easing of international tension and the
strengthening of trust between States in order to facilitate the
cessation of the manufacture of nuclear weapons, the liquidation of
all their existing stockpiles, and the elimination from national
arsenals of nuclear weapons and the means of their delivery pursuant
to a Treaty on general and complete disarmament under strict and
effective international control. [ref]

\subsubsection{Comprehensive Test Ban Treaty (CTBT)}
On 10 September 1996 the treaty was adopted by the UN General
Assembly; on 24 September 1996 it then opened for signature in New
York, when it was signed by 71 States, including five of the eight
then nuclear-capable states. As of January 2012, 156 states have
ratified the CTBT and another 26 states have signed but not ratified
it, including China, Egypt, Iran, Israel and the United States. India,
North Korea and Pakistan have yet to sign it (see Table~\ref{tab:NWSwarheadsCTBT}).

There have been several other treaties with the aim of encouraging
disarmament, with varying degrees of success. Limited success was
achieved with the signing of the Partial Test Ban Treaty in 1963,
which banned nuclear tests in the atmosphere, underwater and in space,
but neither France nor China signed it. However, after an 80 to 19
vote in the United States Senate meant that the treaty was still
ratified by the United States. In 1968 the Nuclear Non-proliferation
Treaty (NPT) was signed, which was a major step towards
non-proliferation of nuclear.  This particular treaty decreed the
prohibition of possessing, manufacturing or acquiring nuclear weapons
or other nuclear explosive devices for non-nuclear weapon states. It
committed all signatories, including nuclear weapon states, to the
goal of total nuclear disarmament. However, as mentioned previously,
India, Pakistan and Israel have declined to ratify the NPT. Their
grounds for doing so were that the treaty is fundamentally
discriminatory; it places limitations on states that do not have
nuclear weapons but does not attempt to restrict weapons development
by those who are declared nuclear weapons states.

Due to the political climate of the next few decades, very little
progress was made in nuclear disarmament until 1991, but that year an
amendment conference was held to discuss the proposal of converting
the Treaty into one banning all nuclear-weapon tests. The UN General
Assembly gave it strong support, leading to negotiations for a
comprehensive test-ban treaty beginning in 1993. It took three years
of intense effort and work to draft the Treaty text and its two
annexes, but a consensus could not be reached on the adoption of the
text in the Conference on Disarmament (in which negotiations were
being held). Under the direction of the Australian Prime Minister John
Howard and Foreign Minister Alexander Downer, the text of a draft
resolution was submitted to the United Nations General Assembly in New
York. The CTBT was finally adopted by a large majority, exceeding
two-thirds of the General Assembly's Membership on 10 September 1996.

U.S. President Barack Obama in April 2010 called for the world's
nuclear weapon arsenal to be vastly reduced. He labelled the thousands
of remaining weapons ``the most dangerous legacy of the Cold War.''
For this to be achieved would require cooperation because the Nuclear
Weapon States (NWS) and a trusted verification and dismantlement
process.

\subsubsection{bits to add in}
Information barriers can be used which would confirm the agreed amount
of radioactive material is correct in the container. These barriers
would be separately built by each country.  Main Objectives Before
dismantlement the total quantities of weapons grade plutonium and
uranium need to be determined for each NWS.  Dismantle process begins
with the warheads being split into their individual components which
include the arming or firing mechanism, the primary physics package,
the secondary physics package.  The warhead needs to be destroyed by
crushing the part until it is rendered militarily useless.
Weapons-grade plutonium is 93.5\% Pu-239 and 6\% Pu-240 whereas
reactor grade plutonium is 58\% Pu-239 and 24\% Pu-240. [ref] Ideal
conditions for a transparent international dismantlement facility
would include a neutron source that could obtain the mass of plutonium
or uranium used to an accuracy of 5\%, and would log this separately
for each weapon.  The facility would ideally be brand new and observed
under construction and shown to have no basements. This means the
facility could be checked for nuclear material before a weapon’s
dismantlement began.  Care is needed when handling hazardous materials
such as beryllium.

\section{Political Background}
\subsection{Treaties}

\subsection{Nick's Stuff}
The Verification process seeks to ensure consistency with declaration,
and that disarmament has been achieved with a degree of
irreversibility.  The challenge facing proponents of Nuclear
disarmament is to ensure that measures have been taken to safeguard
against a Nuclear Weapon State attempting to cheat against the
international agreements that have been put in place. The foundations
for confidence are essentially built upon two key principles:
Declaration and Verification. The interplay between the two is what
establishes the iterative development towards Nuclear disarmament;
because a more complete declaration can lead to greater confidence in
the verification, and successful verification can lead to a decision
to release a more complete declaration. However there is a third
consideration that mediates between the two and also governs what
should be done with the sensitive information that may emerge from the
verification measurements -- the ``information barrier''; the
foundations of which shall be discussed in section ??.

What do we verify? -- Nick’s brief overview Nuclear warhead
dismantlement starts with the removal from warheads from the
deployment areas to the storage areas. The warheads are then
transported to the dismantlement facilities where the ``Physics
package'' is removed and then stored to await further
dismantlement. Lastly plutonium and Highly Enriched Uranium re-emerge
in an unclassified form ready for final disposition.

\begin{figure}
  \includegraphics[width=\linewidth]{pile01/nickpit1.png}
\end{figure}

[dismantlement process.png] [Fig ?? : accompanying text] [Global
  Fissile Material Report 2009: A Path to Nuclear Disarmament, IPFM
  Annual report, Acton et al., p67] Verification may take place at any
point along this chain. A Nuclear Weapon State may declare numbers and
types of weapons in storage sites for example, and it is the job of
the inspector to confirm or refute this information.  However
inspectors do not enjoy the unrestricted right to freely examine the
weapons; they must do so behind the veil of an information barrier
because the act of inspection incurs the risk of spreading weapon
information.

\subsubsection{Information barrier}
In support of several ratified and pending nuclear material control
agreements, technical representatives from the US and Russia have
recognised the necessity for assurances against the release of
sensitive information to be put in place.  The majority of these
agreements involve storing nuclear materials and components from
stockpile weapons within specially designed containers.

Strategies for monitoring the agreements include measuring the neutron
and gamma radiation signature to verify declared attributes of the
plutonium or HEU.  If these measurements are accurate enough to serve
for this verification purpose, then they are accurate enough to
contain information about the design of the component being
monitored. Subsequently safeguards have been designed to prevent the
disclosure of that information. Hardware, software and procedural
measures containing the sensitive data will only present the relevant
results required for verification.

[Progress in Gamma Ray Measurement Information Barriers for Nuclear
  Material Transparency Monitoring'', Wolford and White, Lawrence
  Livermore National Laboratory Library, 2000] In the interest of
transparent monitoring, inspectors may witness or perform restricted
measurements on controlled items. The information barrier is designed
to mitigate the intrusiveness of taking these measurements, and the
proliferation knowledge it would pass on to the inspector. Wolford and
White (2000) highlight three objectives that hardware, software, and
human procedures should fulfil in order to be an effective
information barrier: ``Prevent the unintended release of sensitive
information during an inspection; Display a simple but reliable and
useful result to the inspector; Allow checks on the integrity of the
internal operations not visible during an inspection.''  However in
applying the barrier, the person monitoring may lose the assurance
that the internal operations proceeded as intended. Fortunately it can
be shown that thoughtful design elements can help recover some of that
lost assurance for the human operator.

\subsubsection{Design Elements}
An actual information barrier must be adapted to the measurement
instrument it accompanies. However there are certain design elements
that must be implemented no matter what type of measurement is being
performed. The DOE-DOD information Barrier Working Group has provided
guidance for 10 design bases which, when grouped into functional
categories, fall into 3 top-level elements: ``(1) A barrier to conceal
the sensitive information gathered in a measurement, and from which
the physical attributes of an inspected item are derived. This
consists of some combination of hardware, software, and human
procedures, and must work in both directions, shielding unintended
signals originating both outside and inside the measurement system.
(2) A simplified display that indicates clearly the selected results
of the measurements as defined in the agreement, and nothing
more. Accordingly, the display should be no more complex than is
necessary to convey the result to the inspector.  (3) Enough autonomy
to compensate for the lack of a human operator, both in monitoring the
measurement and in safeguarding the data. The instrument must assure
the reliability of its own measurements as well as protect the data
resident during an inspection. In the event of failure or signs of
tampering, this mechanisms should erase all traces of sensitive data
from the instrument and halt the inspection.''  [ THE JOINT
  U.S. DOD-DOE INFORMATION BARRIER WORKING GROUP, Functional
  Requirements for Information Barriers, PNNL-13285, Pacific Northwest
  National Laboratory, Richland, WA, May 1999.]

How do we verify? -- Nick’s Pit stuffing `Pit-stuffing' was developed
at Los Alamos National Laboratory to ensure that warheads that had
been internally evaluated to be unsafe would not accidentally go
off. Pit-Stuffing makes it possible to disable thousands of nuclear
warheads, quickly, cheaply, and irreversibly; in a verifiable manner.

\begin{figure}
  \includegraphics[width=\linewidth]{pile01/nickpit2.png}
\end{figure}

[Pits.png Fig?? ] Figure 5.3. Storage arrangements for U.S. plutonium
warhead ``pits'' at the Pantex warhead dismantlement facility in
Amarillo, Texas.217 [Global Fissile Material Report 2009: A Path to
  Nuclear Disarmament, IPFM Annual report, Acton et al., p71] Modern
implosion-type nuclear weapons have a ``pit'' which is a hollow sphere
of plutonium or highly-enriched uranium, with a tiny tube through
which the tritium is passed into the hollow sphere. If an appropriate
material is fed in through this small tube until the inside of the pit
is ‘stuffed’, and the plutonium can no longer be compressed enough to
reach the critical mass to sustain a nuclear chain reaction as it
would encounter the fill on the way.

At Los Alamos there had been discussions as to what would be the most
suitable material. Aluminium powder and epoxy were suggested, however
powder could be made to fall back out the tube, and organic material
within the vicinity of plutonium could trigger chemical radiations and
thus threaten the safety.

The best considered option is to use bits of metal wire that are
shaped so that they cannot be removed via the fill tube.  If this were
achieved the only way to make the weapon functional again would be to
dismantle it, remove the pit and cut it open to take the wire out;
then re-manufacture the pit and reassemble the weapon. This would be an
expensive process, especially compared to the minimal time it takes to
fill the pit.  The physical act of stuffing the pit would about a
minute; therefore a single inspection visit could be very productive,
even if additional time is spent carrying out necessary safety
procedures.  Verifying that the pit had indeed been filled could be
confirmed by incorporating micro-curie quantities of cobalt-60 in the
stuffing wire. One set of gamma-ray counters aligned to view the pit
from one side would give a few simultaneous counts with another
gamma-ray detector orientated in a perpendicular direction.  This is
because cobalt-60 gives two simultaneous high-energy gamma rays. Such
measurements could not be mimicked by gamma-ray sources that are not
in the interior of the pit.  [R. Garwin, Technologies and procedures
  for verifying warhead status and dismantlement, Transparency in
  Nuclear Warheads and Materials: The Political and Technical
  Dimensions edited by Nicholas Zarimpas, 2001] However further
measures may have to be taken against the host merely inserting
minimal amounts of cobalt-60 without inserting the wire, or
semi-inserting the wire with a view to remove it after the
inspection. Also because pit-stuffing depends on certain details of
warhead design and fabrication, it might not an approach that can be
administered entirely blindly.

The practicalities of this approach are encouraging for the cause of
verified dismantlement because of the speed and minimal cost. The
intrusiveness in principle could be low, if there was assurance that
the wire had been inserted and could not be removed. The host could
then be free to complete the dismantlement in privacy, removing the
need for foreign verification during transportation which is currently
a considerable expense.

After the dismantlement, the inspectors would return and be shown the
canisters containing the stuffed pits. Again a gamma-ray spectrum
could confirm that the containers enclosed hollow spheres of plutonium
stuffed with wire. The inspector could be very confident that these
were the same pits observed before the dismantlement as it would be
very expensive and cumbersome to manufacture thousands of hollow
plutonium spheres stuffed with wire.

[`Pit-Stuffing': How to Disable Thousands of Warheads and Easily
  Verify Their Dismantlement, Bunn, Federation of American Scientists
  Public Interest Report, volume 51, issue 2, pages 3-5, 1998] This
method would work for US warheads, but needs to be evaluated for
Russian warheads that may have a different design.  [R. Garwin,
  Technologies and procedures for verifying warhead status and
  dismantlement, Transparency in Nuclear Warheads and Materials: The
  Political and Technical Dimensions edited by Nicholas Zarimpas,
  2001] However pit-stuffing as a verification technique might not get
it’s opportunity to be utilised because of the asymmetry between the
sensitivity of classified items between Russia and the US. The years
2000-2002 saw dramatic global change, including the changing of
American and Russian leaders. This meant that the enthusiasm for
implementing Initiatives negotiated in the late 1990s was not
sustained. The Russian Federation decided that it would melt its pits
into 2-kilogram balls and pack two plutonium balls into each
specialized AT-400R container before submitting the material for
U.S. or IAEA verification. The US and Japan provided AT-400R's are the
standard containers designed for Russia's Mayak Fissile Material
Storage facility. Russia maintained that the isotopic composition of
its weapon plutonium in the 2-kg balls was classified.  [Global
  Fissile Material Report 2008, p. 70]

Although political decisions on declarations might stand in the way of
pit-stuffing as a universal verification technique, further work*
could be done to address some of the technical issues discussed as it
could prove to be a useful tool to verify the US.

*(perhaps similar to the UK Norway Initiative Workshop on Nuclear
Disarmament Verification 7-9 December 2011) Example: Information
Barrier for Gamma Ray Measurements Gamma ray measurements can infer a
lot of information about the object being measured including,
constituents abundances, a lower bound for the masses, and the amount
of intervening material. Also in a neutron-emitting source such as
plutonium, the presence of other elements can be inferred from the
evidence of activation products. Clearly much of this information lies
outside that demanded of transparency agreements and fortunately a
full spectrum is not required to derive the most useful attributes.

Programmers have an opportunity to protect most of the information. At
LLNL software was created that used plutonium lines between 630 KeV
and 670 KeV to compute a ratio of 240Pu to 239Pu which distinguishes
weapon-grade plutonium from non-weapon grade. The tool, called Pu600,
was adapted and enhanced for the requirements of the Trilateral
Initiative and the FMTT program. Programmers at LLNL developed similar
methods called Pu300 for determining the time since separation, and
Pu900 for determining the amount of oxide present in a sample.

\begin{figure}
  \includegraphics[width=\linewidth]{pile01/nickpit3.png}
\end{figure}

[Progress in Gamma Ray Measurement Information Barriers for Nuclear
  Material Transparency Monitoring", Wolford and White, Lawrence
  Livermore National Laboratory Library, 2000] [Fig??: The gamma ray
  spectrum for a non-sensitive sample of plutonium containing several
  isotopes and decay products. The three narrow intervals of spectral
  lines are what is used in the attribute calculations Pu 300, Pu600
  and Pu900.]

\subsubsection{Ideas for conclusion}
The United States has stated the need to protect the host's warhead
design information overrules the need to provide confidence to the
inspecting party regarding the accuracy and reproducibility of the
measurements; in case an inspection is carried out in the absence of
any agreement to share classified nuclear weapon design.

We could recommend that this shouldn't be the case for Non-Nuclear
weapons states like Iran, where their proliferation information is
obviously less developed than that of the US or Russia.  Talking about
info barrier.. The greatest potential for improvement in the
transition from generic laboratory instrumentation to more special
purpose inspection equipment, and the greatest challenge is to move
away from stored programs and microprocessors towards pure hardware
solutions.  The technology of information barriers is primitive
compared to the technology of radiation detection and data
reduction. Nevertheless, relatively simple systems such as those
described above and in the references have provided sufficient
assurance to sustain negotiations. The specific implementation of an
information barrier system will depend on the requirements of the
inspection regime. Nevertheless the three elements introduced here
will always form its basis. More specific design influences include
decisions about equipment origin and custody and the number and type
of physical attributes to be collected.  [Progress in Gamma Ray
  Measurement Information Barriers for Nuclear Material Transparency
  Monitoring'', Wolford and White, Lawrence Livermore National
  Laboratory Library, 2000]

The treaty should challenge the Host facility to demonstrate
compliance: Change the emphasis of the verification
procedure. Consider that a Nuclear Weapon State intends to cheat the
disarmament agreements that are in place.  ``Winning'' if them is if
they manage to fool the inspector. However if the onus were on host,
it would be their duty to convince the inspector they are abiding by
the agreements; for the cheat ``winning'' would be managing to
convince the inspector. This approach addresses the underlying
psychology between inspector and host facility more positively.

\section{Overall Process}
\subsection{Dismantlement}
Joint exercise between UK and Norway describes a possible
dismantlement and verification process, conducted near Oslo in
2009. Cobalt 60 sample used to mimic physics package. A single nuclear
weapon was moved into a dismantlement facility, where screwdrivers are
used to open a side panel and remove the physics package. It is placed
in a separate container. Everyone entering or leaving the
dismantlement room is checked for radioactive material. Inspectors
then confirm that radioactive material is present in the container. It
is stored overnight in a sealed room with CCTV. Seals used on items
are low tech. The doors on the overnight storage room had a purple
adhesive down the hinge which would change colour if disturbed. The
adhesive strip itself is identifiable with a cluster of glitter
suspended in transparent glue. On application and removal of the
adhesive strip, this glue tag is photographed to check it has not been
replaced, the exact configuration of suspended pieces of glitter would
be near impossible to replicate. A simple technology like this seems
to provide a reliable way of detecting tampering with a door or
package seal. The following day the physics package would be moved to
the ``hot cell'' where it would be dismantled.  Using an ``information
barrier'' device, the amount of material in the package can be
measured without revealing the quantity to the inspectors. It gives a
green light if the result was the same as the last measurement. Both
countries independently developed and built their own detector. Using
this method the package can be inspected at multiply points to insure
no material has been siphoned off. Once the weapon has been dismantled
the nuclear material is taken away for storage.[1]

\begin{figure}
  \includegraphics[width=\linewidth]{pile01/ralph1.png}
\end{figure}

The magazine transparency system is a proposed method of storage to
monitor and maintain the inventory of a ``magazine'', a storage area
containing nuclear warheads or nuclear fissile materials during
dismantlement verification operations.

\begin{figure}
  \includegraphics[width=\linewidth]{pile01/ralph2.png}
  \caption{A schematic of the Magazine Transparency System (MTS) 
    showing the primary components.}
\end{figure}

A system like this has been demonstrated at the Pantex dismantlement
facility in Texas. The weapon itself has only passive tags and seals
placed upon it to eliminate problems associated with battery life. The
MagTag blanket is simply a tarp containing permanent magnets in random
orientations, so as to make it unique. The system has a high
resolution magnetometer to detect changes in the position or
characteristics of the blanket. The bar-code reader records the
bar-code on the seal of each container when the magazine is emptied or
filled. this data is stored on an integrated notebook computer which
can transmit to a central monitoring station. The RF receiver
transmits the unique RF tag number on each weapon/storage vessel to
the notebook. Low light video cameras also send video straight to the
notebook for processing. With all of this information, Gauss readings,
video, RF tag etc... the notebook computer sends an ``all okay'' signal
once every second assuming no anomalous activity is detected from any
of the sensors. The notebook computer records no Non Proliferation
treaty sensitive information.  In the event of anomalous activity in
an MTS during disarmament verification operations, the notebook is
immediately given to the inspectors who can determine what kind of
activity took place. [2]

Many weapons are dismantled without foreign verification by nuclear
weapon states.  This is done to reduce the size of the stockpile or to
dispose of older outdated weapons.  In the US, the national nuclear
security administration (NNSA) is ``responsible for the management and
security of the nation's nuclear weapons, nuclear nonproliferation,
and naval reactor programs'' [2] Part of this management is weapon
dismantlement. design laboratory's work with production facilities to
identify and mitigate any risks associated with dismantling a
particular design of weapon. A plan is formulated to safely dismantle
the weapon and it is taken to the Pantex Plant in Texas. The time
required to dismantle a warhead ranges from ``a few days to a few
weeks'' depending on the complexity of the design.[3] With the weapon
dismantled, the high explosives and other non-nuclear materials are
processed on site at Pantex and a few other facilities.  The special
nuclear materials are then dismantled at the Y-12 National Security
Complex.  [4] Here nuclear materials are down blended to reduce the
enrichment and render the uranium useless for military
application. Down blended uranium is then used to fuel nuclear power
plants. According to the department of energy 10\% of the electricity
used in the United States is produced using former Russian nuclear
weapons.[5] Uranium ``pits'' that are not down blended can be stored
in a ``sealed insert'' system. Leak-tight stainless steal vessels are
placed into steel storage over pack drums.

\subsubsection{Downblending Uranium}
Downblending is the opposite of enrichment; excess Highly-Enriched
Uranium (HEU) can be downblended to Low-Enriched Uranium (LEU), which
can then be used commercially in nuclear fuel. Natural uranium is
comprised of three main isotopes, U-234 which is 0.01\% by weight,
U-235 which is 0.71\% by weight and U-238 which is 99.28\% by weight.
Of the three naturally occurring constituents U-235 is the only
fissile isotope.  Uranium that is enriched enough to be used in
nuclear weapons is typically \(>\)90\% Uranium-235 whereas in
commercial reactors this enrichment is 3-5\% Uranium-235.  HEU is
defined by having a U-235 content of greater than 20\%. HEU can only
be used in nuclear weapons and in research reactors.

The surplus of HEU from dismantled nuclear weapons can be downblended
to LEU so that in can be used in commercial nuclear power plants. An
example of this is the Megatons to Megawatts Program converts
ex-Soviet weapons-grade HEU to fuel for U.S. commercial power
reactors. From 1995 through mid-2005, 250 tonnes of high-enriched
uranium (enough for 10\,000 warheads) was recycled LEU. The goal is to
recycle 500 tonnes by 2013. The decommissioning programme of Russian
nuclear warheads accounted for about 13\% of total world requirement
for enriched uranium leading up to 2008 [ref]. The United States
Enrichment Corporation has been involved in the disposition of a
portion of the 174.3 tonnes of highly enriched uranium (HEU) that the
U.S. government declared as surplus military material in 1996. Through
the U.S. HEU Downblending Program, this HEU material, taken primarily
from dismantled U.S. nuclear warheads, was recycled into low-enriched
uranium (LEU) fuel, used by nuclear power plants to generate
electricity [ref]

During the downblending process, HEU is needed to be blended to a
concentration of 3-5\% U-235 so that it is then LEU and can be used as
nuclear reactor fuel. For this purpose, there are two techniques that
can be implemented -- blending as uranyl nitrate solution and blending
as uranium hexafluoride.

Converting uranium into uranyl nitrate is considered appropriate to
produce LEU material.  When uranium is dissolved in nitric acid, an
aqueous solution of uranyl nitrate
[\(\mathrm{UO_2(NO_3)_2\,.\,}x\mathrm{H_2O}\)] is formed. The
preferred uranium compound used for this is \(\mathrm{U_3O_8}\) in
powder form, as it has a high surface-area-to-volume ratio, thus
enhancing dissolution. The rate at which uranium is fed into the
process can be monitored and metered allowing the whole reaction to be
well regulated.

This process can produce a wide variety of compounds of purified
uranium. When LEU is required for waste disposal by way of
downblending procedures, the mixed LEU UN solution is converted to
\(\mathrm{U_3O_8}\). The concentrated LEU UN solution is then
thermally decomposed to produce \(\mathrm{UO_3}\), which can then be
oxidised producing impure LEU \(\mathrm{U_3O_8}\) powder.

The downblended LEU produced (either UNH crystal or
\(\mathrm{U_3O_8}\)) is packaged and delivered to either a reactor
fuel fabrication facility or a LLW disposal site. In the United States
the U.S. Department of Transportation (DOT) allows these LEU compounds
to be shipped overland in approved packaging containers.

Uranium into uranium hexafluoride is the other process achievable for
downblending.  The \(\mathrm{UF_6}\) blending process described here
is favoured for diluting the U-235 isotope abundance in HEU being
blended for use as reactor fuel.

If subsequent blending operations are to be performed, any excess HEU
feed material needs to be converted to \(\mathrm{UF_6}\), as most of
the surplus HEU being downblended for use as reactor fuel will be
either in oxide compounds or metal form. No fluorination process is
required for the diluent stream, as low-concentration
\(\mathrm{UF_6}\) can easily be made available from previously
existing inventories and domestic facilities.

In order to convert uranium metal and oxide compounds into
\(\mathrm{UF_6}\), many methods can be employed. One such method
converts uses two separate hydrofluorination and fluorination
reactions to convert \(\mathrm{UO_2}\) into
\(\mathrm{UF_6}\). Provided they are first processed for conversion to
\(\mathrm{UO_2}\), this method can also be used for uranium metal and
\(\mathrm{UO_3}\).

Hydrofluorination is the first step of this process, and it occurs in
a fluid bed reactor where hydrogen fluoride (HF) is reacted with the
\(\mathrm{UO_2}\) resulting in the production of uranium tetrafluoride
\(\mathrm{(UF_4)}\):
\[ \mathrm{ UO_2 + 4HF \rightarrow UF_4 + 2 H_2O } \]
Fluorination is the second step, and this occurs in a vertical tower
reactor where fluorine \(\mathrm{(F_2)}\) is reacted with
\(\mathrm{UF_4}\), producing \(\mathrm{UF_6}\): 
\[ \mathrm{ UF_4 + F_2 \rightarrow UF_6 } \]
There is a series of cold `traps' in the equipment for capturing the
\(\mathrm{UF_6}\) gas and it can be sublimed into a solid. Any
gaseous \(\mathrm{UF_6}\) that gets past these cold traps is caught by
sodium fluoride (NaF) trap.

Both hydrofluorination and fluorination produce exhaust substances,
which are passed through a further series of filters in order to trap
any remaining HEU material. The substances passing through these
filters is carefully monitored to verify the absence of uranium and in
order to prevent any HF or \(\mathrm{F_2}\) from being released the
exhaust substances are then scrubbed with potassium hydroxide (KOH).

There is also a single, direct fluorination method that allows to
conversion of Uranium metal into \(\mathrm{UF_6}\). The uranium metal
is sealed in a tube furnace with a gaseous mixture of \(\mathrm{F_2}\)
and a diluent such as nitrogen. The heat from the furnace initiates
the fluorination reaction, but excess \(\mathrm{F_2}\) must be added
for the reaction temperature to be sustained: 
\[ \mathrm{ U + 2F_2 \rightarrow UF_4 } \]
The \(\mathrm{UF_6}\) gas is then collected and the exhaust gases filtered as in the two-step 
hydrofluorination/fluorination method described previously. [ref]

\subsubsection{Storage of Fissile materials/nuclear weapons}
Assuming confidence in the dismantlement of a warhead, for real
progress to made storage is equally important. Nuclear weapon fissile
materials need to be stored under international or bilateral
monitoring, to prevent the construction of new weapons and also
safeguard against theft.

\subsubsection{MPC+A, Material Protection Control and Accounting}
This program is an example of bilateral efforts to safely and securely
store nuclear materials. Funded initially by the DOD and then by the
DOE, its goal was to improve security on hundreds of buildings at 40
sites in Russia to prevent theft or loss of nuclear materials. 600
tons of material was found to be at ``risk''. in 2001, a report
detailed that so far 81 buildings housing 81 tons of fissile material
had received upgraded security systems. In 1998 the program was
predicted to finish upgrading the storage facility's by 2020. [6]

\subsection{Chain of Custody}
Chain of custody refers to the system of routines used to provide a
high confidence level that a nuclear warhead will be delivered from
its field deployment or storage location to the dismantlement facility
and that the recovered nuclear material will be monitored until its
disposition to make sure that it will not be reused
\citep{zarimpas2003}. Chain of custody includes the use of photography
(X-ray or gamma ray images, optical pictures), seals and tags, visual
surveillance or video, warhead authentication (detection and
confirmation of nuclear or non-nuclear warhead) and information
processing (data transmission and relay, data compression, encryption
and decryption etc.). There are two categories of chain of custody:
limited chain of custody of a specific dismantlement facility and full
range chain of custody with continuous knowledge of the disarmament
process and the total number of nuclear warheads. \citep{wuwen2004}

\subsubsection{Seals and Tags}
A tag is a unique characteristic (`fingerprint') of an object or
container that is used to unambiguously identify it. A seal or
tamper-indicating device is a device or material that records
unauthorised access and leaves inerasable evidence behind. There are
two categories of modern seals: passive and active. Passive seals do
not need electrical power to work and they are inexpensive but they
can only be used once. Active or dynamic seals require electrical
power to work, either internally or externally. They are more
expensive than passive seals but they are typically reusable. [3]

Seals used in verification of nuclear disarmament need some unique
attributes such as transparency and negotiability. If the inspected
facility provides and controls the seals, then the inspectors will be
suspicious and think that the seals have been tampered with. If the
inspectors provide and control the seals, then the inspected facility
will worry about espionage devices such as microphones being embedded
in the seals. High tech electronic seals could be less comfortable to
both parties due to tampering, safety and espionage concerns. Also
some weapon containers cannot accommodate extra seals so the seals for
nuclear disarmament will have to coexist with seals for safeguards and
internal security. [3]

Current technologies and technologies under development:
\citep{zarimpas2003} Reflective Particle Tags (RPT): Reflective
particles are mixed into glue that is applied to the object being
tagged. If the tag is illuminated by a point light source, particle
reflections form a unique pattern to that tag and incident angle. A
tag reader records the pattern of individual pictures using a camera
and forms a set of fingerprints unique to that tag. The object can be
uniquely identified by reading its tag. [4] Ultrasonic Intrinsic Tags
(UIT): A sample is scanned ultrasonically and a hand-held scanner
collects sub-surface structure data. A computer performs the alignment
and correlation functions. These tags are resistant to surface changes
and counterfeit.  Surface feature tags: A unique fingerprint of an
item by examining its surface using scanning electron microscopy,
holographic interferometry and micro-videography.  Shrink-wrap seals:
They consist of a plastic film that shrinks tightly around the
safeguarded object. A unique pattern is produced by multiple layers of
geometrically patterned film and it’s photographed for verification.
VACOSS fibre optic seal: The first IAEA electronic seal which consists
of a loop of fibre optic cable that is actively checked for integrity
by the seal’s electronic system. Remote reading of the seal is also
possible. They are no longer supported by the manufacturer and are
being replaced by the EOSS seal which is more secure. [5] Electronic
Optical Sealing System (EOSS): It is intended for long duration
surveillance with high reliability. It uses a seal for enhanced
authentication, smart power management system, tamper indication and
encryption. A tamper indicating enclosure protects the electronics and
a microwave foil protects the complete housing against drilling. It
can be coupled with surveillance systems and can be remotely
interrogated. [5] Cobra seals: Consist of a loop of fibre optic cable
and a polycarbonate sealing body. A unique light pattern is created
when a blade cuts the cable. The Cobra seal reader photographically
records this unique pattern and uses it for future comparison.  E-type
cup wire loop seals: Consist of two metal cups which snap together and
cover the crimped ends of a wire loop. The inside of the cups is
covered with melted solder and is scratched to create a unique pattern
that is recorded for future comparisons.  E-tag mechanical seals:
Similar to E-type cup wire loop seals but also contain an electronic
chip that contains a unique identification number that can be verified
without opening the seal.  Pressure-sensitive adhesive seals: Use
pressure-sensitive adhesives to attach fragile labels to an
object. Unique reflective patterns are created by microscopic glass
beads.  T-1 radio-frequency seals and tags: Consist of a fibre optic
seal, case tamper switches, motion detector and low and high
temperature indicators.  Acoustic tags: Based on the unique resonant
properties of an object when scanned using sound waves of a specific
frequency.  Radio-frequency (RF) tags: Emit a unique identification
number and can be scanned using a radio-frequency device.  VNIIEF
smart bolts: A small reader is used to read the unique electrical
properties and the digital identification of the seal. The electrical
properties of the bolt change if it’s unscrewed and that indicates
tampering.

\subsubsection{Vulnerability of seals}
All seals appear to be vulnerable to simple, rapid, low tech
attacks. Attacking a seal means to trying to gain access to whatever
the seal is protecting without being detected. Defeating a seal means
opening the seal without being detected or leaving evidence of entry,
or repairing any damage and erasing evidence of entry, or replacing
the entire seal or parts of it with counterfeits. The most
comprehensive seal vulnerability study has been undertaken by the
Vulnerability Assessment Team (VAT) at the Los Alamos National
Laboratory. They analysed 94 different seals and defeated them for a
total of 132 defeats. All of the defeats were implemented using low
tech attacks with tools and supplies that in some cases can be easily
carried in a person’s pockets. The time taken for successful attacks
varied from 3 seconds for several seals to 125 min for the most
difficult seals. The mean time taken to complete the 132 defeats was
4.3 min and the mean cost of the defeats was \$56. [6]

\subsubsection{Surveillance technologies}
Surveillance is another important element in the chain of custody and
comprehensive surveillance can also increase the security status of
the monitored nuclear warheads. Comprehensive surveillance provides a
high confidence in the dismantlement process and establishes mutual
trust between the involved countries. Comprehensive surveillance
combined with seals and tags can constitute an in depth inspection
system. Several technologies can be used for comprehensive
surveillance such as video monitoring, sensor monitoring, photograph
comparing, intrusion detection and satellite
imaging. \citep{wuwen2004}

Video monitoring is widely used in international safeguards and
domestic security and it plays an important role in preventing illegal
activities and proving treaty compliant activities. It is one of the
most widely used containment and surveillance techniques by the
IAEA. \citep{wuwen2004} Modern digital imaging used by the IAEA
includes single camera digital surveillance units such as the
All-in-one System (ALIS), All-in-one Portable System (ALIP), Digital
Single Camera Optical Surveillance System (DSOS) and multi-camera
digital surveillance systems such as the Server Digital Image
Surveillance System (SDIS) and the Digital Multi-camera Optical
Surveillance System (DMOS). These systems were developed to replace
aging videotape based systems and the SDIS can be used for remote and
unattended operation. The General Advanced Review Station Software
(GARS) was developed to review all digital image surveillance records
of the IAEA using a desktop computer. [7] The newest system is the
Next Generation of Surveillance System (NGSS) and it will be used in
2012. It is scalable to any number of cameras, has solid state storage
media, low power consumption, advanced security features and is highly
reliable under harsh environmental conditions. It can be configured as
a single all in one camera system or as a scalable multi-camera
system. It supports various trigger signals from electronic seals and
sensors, high resolution and coloured images, remote monitoring and
picture taking rates of one image per second. Four different fields of
view can be recorded simultaneously by a single NGSS camera. [5]

\begin{figure}
  \includegraphics[width=\linewidth]{pile01/CoC1.png}
  \caption{Picture of the Next Generation of Surveillance System
    (NGSS) with 24 cameras [5].}
\end{figure}

Sensor monitoring is another effective method used to track and
authenticate nuclear warheads.  Various sensors such as
motion/acceleration, weight, acoustic, radioactive, magnetic, thermal,
temperature/humidity, impact force sensitive and vibration sensors can
be combined to monitor nuclear warhead reduction and provide a high
confidence. Most of these sensors are commercially available and can
be used in an integrated monitoring system which is easy to
understand, technically transparent and applicable by all
parties. \citep{wuwen2004}

Photograph comparing is used to carefully examine recorded images of
the objects being monitored and detect tamper activities. Drawing
random marks or images near the monitored object and then photograph
it from all angles can enhance the photograph comparing method. It can
easily raise disputes because it requires specific interpretation,
either algorithmic or artificial, and the picture taking process can
be also affected by environmental elements so it is never used on its
own. It is often embedded into tamper-indicating monitoring methods
and used as a complimentary technique in nuclear
disarmament. \citep{wuwen2004}

Intrusion detection is used in guarding nuclear warheads and uses
microwave, capacity sensitive, weight detection, infrared beam and
motion/acceleration sensors to detect unauthorised access.  The
technologies and intrusion detection methods are being continuously
developed to increase the efficiency and reliability of intrusion
detection. \citep{wuwen2004}

Satellite imaging is used to detect change of large facilities or
objects and requires advanced space and electronic sensor
technologies. It can integrate visual, multi-spectra, synthesised
aperture radar (SAR), thermal/infrared and other photographic
technologies into a comprehensive system to provide a meaning full
picture of a specific area or object. \citep{wuwen2004}

\subsubsection{Vulnerability in chain of custody}
The Vulnerability Assessment Team (VAT) at the Los Alamos National
Laboratory has extensively researched issues associated with nuclear
safeguards in the areas of transport security, intrusion detection and
vulnerability assessments.

Most organisations ignore or underestimate the insider threat which
the security risk due to its employees. [8] Motivations for an insider
attack include revenge, greed, ideology, terrorism, social engineering
or even mental illness and periodic background checks should be used
to reduce the insider attacks. Common mistakes that organisations make
regarding the insider threat are thinking that low-level employees are
not a threat, not testing if employees can be bribed, thinking that
only employees are insiders and having overconfidence in the polygraph
test. [9] The IAEA does little or no background check on its employees
either before or after hiring, including nuclear inspectors. The IAEA
must trust its inspector’s judgement on whether treaty violations are
occurring and the lack of background checks puts the reliability of
inspections at risk. Also IAEA inspectors are granted diplomatic
privileges and the position of an inspector could attract people who
want to exploit diplomatic status for terrorist or criminal
activities. [8]

Another security risk is the fact that people are poor observers and
they don't realise it. Perceptual blindness, also known as
inattentional blindness, is the phenomenon of not being able to
perceive things which are in plain sight if you are focused on a
specific visual task. Change blindness is also a kind of perceptual
blindness is the phenomenon of observers failing to notice changes,
including blatant changes, even when they are expected. This has
serious implications for inspectors and security guards who inspect
seals, watch video monitors, operate safeguards equipment and guard
gates. Inspectors and security guards should be educated and trained
on this and technology could be used to cover for perceptual
weaknesses in humans. [9]

Confusing inventory with security leads to bad security. Inventory is
counting and locating stuff and it can detect innocent errors by
insiders such as sending a shipment to the wrong location, but it’s
not designed to detect spoofing and deliberate nefarious attacks by
insiders or outsiders. [10] Security is meant to counter nefarious
adversaries, both insiders and outsiders. [9] Inventory systems should
not be used as security systems. Examples of inventory systems used as
security systems in nuclear safeguards include using the Global
Positioning System (GPS) for cargo security and using contact memory
buttons (CMBs) or radio frequency identification devices (RFIDs) for
nuclear material control and accounting (MC\&A). [10] Almost all
nuclear applications use the civilian GPS signals and not the military
signals.  The civilian GPS signals are unauthenticated, unencrypted so
they are not secure and were never meant to be used for security
applications. GPS receivers can be easily spoofed or jammed using
widely available commercial GPS satellite simulators which can be
easily purchased, rented or stolen. CMBs and RFIDs are very useful for
inventory but they are not designed to deal with nefarious
adversaries. They are very easy to lift, counterfeit, tamper with the
reader or spoof the reader from a distance. [8] On the other hand
nuclear MC\&A may look like an inventory function since it involves
counting and locating nuclear assets but it is a security function
meant to detect nuclear theft, tampering, diversion or espionage.  A
nuclear MC\&A program or system should make significant attempts to
deter or detect spoofing by a nefarious adversary. [10]

\subsubsection{Unconventional approaches to chain of custody}

Conventional tamper detection methods are fundamentally flawed. A
conventional seal must store the information indicating that it has
been opened until it is inspected but it’s too easy to hide or erase
this information or even make a counterfeit seal. A much better
approach is to use the ``anti-evidence'' method. Information that
tampering has not occurred (the anti-evidence) is stored at the
beginning when the seal is first installed. [8] If tampering is
detected, this information, which is usually 1 byte in length for
electronic anti-evidence seals, is being erased [11].  The inspector
will look for anti-evidence during inspection and it is missing or it
is incorrect then tampering has occurred. If the anti-evidence is
intact then the seal has not been opened. Only the people who
installed the seal know the anti-evidence information which is
different for each seal and changes if the seal is reused. The
attacker will not know the anti-evidence information so he cannot
counterfeit, erase or hide anything. [8] It is also theoretically
possible for the inspected party to check the anti-evidence seal and
send the anti-evidence to the inspectors without them being present
there. [11] This is very useful because inspectors might not be
allowed to handle nuclear material or weapons and might be limited to
observing the facility personnel installing or removing the seals. An
example of an anti-evidence seal is the Time Trap, shown in
Figure~\ref{fig:CoC2}, which can be placed on the hasp of a door or
container. It turns on its liquid crystal display when entry has
occurred and the display will alternate between the time that the
entry has occurred and the hash for that time. Each time has a
different hash and only the people who installed it know the correct
hash for future times. When the seal detects entry, the future hash
values are instantly erased so the attackers cannot counterfeit the
seal because they don't know what hash it should display. [8]

\begin{figure}
  \includegraphics[width=\linewidth]{pile01/CoC2.png}
  \caption{Working Time Trap prototype showing entry occurring at
    13:22 on 3\textsuperscript{rd} February 2005. The picture on the
    left shows the time that intrusion was detected and the picture on
    the right shows the hash for that time. If the hash value is wrong
    or missing or the time is different by a few minutes, then
    tampering has occurred. An on-board programmable microprocessor
    controls this anti-evidence seal and the entire device cost
    \$8. [8]}
  \label{fig:CoC2}
\end{figure}

The anti-evidence method can be used for real-time monitoring of
nuclear material, including during transport and this approach is
called the ``Town Crier'' method. It does not send out an alarm when
intrusion is detected because the alarm signal can be easily blocked
or jammed. Instead it sends a periodic ``ALL OK'' byte called the
bingo number. Only the monitoring system and the people listening in
know the correct bingo number at any given time and if the bingo
number fails to arrive on time, it indicates trouble. The probability
of guessing one bingo number correctly is 1/256 (0.4\%) and the
probability of guessing two correctly is 1/65536 (0.002\%) so the
attackers cannot counterfeit the bingo number. It is a simple, low
cost method that uses a very low bandwidth of 1 byte per minute,
one-way communication and provides high security since attackers gain
nothing by blocking the signal. [11]

Another method that could be used is colorimetry because colour is very
difficult property to accurately reproduce. Inexpensive commercial
colour sensors can measure colour accurately and they can be used as
change detectors for seals. Inspectors could also use them to check if
the walls of a camera enclosure have been cut open, then repaired and
repainted to hide the intrusion. [11]

\section{Detection Schemes}
\subsection{Passive Detection}

Nuclear warhead detectability is reliant upon 4 factors. These factors
are: the warhead design, the technique used for detection, the
sensitivity of the detectors, and any material between the warhead and
detector that functions as a shield for the radiation. [1]

In order for detectors to accurately identify special nuclear material
(SNM), they need to be able to accurately isolate the true signal (for
example, the gamma ray spectrum) from noise (background
radiation). That is, the detector needs to have a high signal-to-noise
ratio. A detector’s ability to obtain a true signal depends on its
efficiency and spectral resolution.[2]

The efficiency of a detector is a measure of its ability to detect
radiation and the rate at which the detector is able to record
data. [2][3]. SNM emits radiation in all directions; however its
intensity decreases with distance. Therefore, to improve efficiency,
the detector should be large or close to the SNM. It should also
record data as fast as possible so as to reduce the time spent waiting
for the scan to take place. [2]

Spectral resolution describes the sharpness of peaks in a gamma ray
spectrum. Since a radioactive isotope releases radiation at certain
energies, an ideal detector would record the spectrum as vertical
lines corresponding to those energies. However, detectors are not
ideal, thus the energy is recorded as a bell shaped curve. The closer
the curve is to a vertical line, the more beneficial the data is. [2]

The features of materials which allow them to be detected or
identified are: 1. Atomic Number and Density: The atomic number (Z) is
the number of protons in the nucleus of the atom; it is unique to
every element. Density is described as mass per unit volume. Plutonium
has an atomic number which is slightly higher than uranium but its
density varies from slightly more to slightly less than uranium [2].
2. Radioactivity: Atoms which are radioactive are unstable and give of
different types of radiation in order to become stable. The types of
radiation used in nuclear detection are gamma rays and neutrons
[2]. Gamma ray spectra are unique to each isotope, thus, it is fairly
easy to determine the radiation that is emitted from an isotope.
3. Photon Opacity: The impenetrability of a material to a photon beam
depends on three things: the amount of material presented to the path
of the beam, the energy of the photons and the atomic number and
density of the material. Gamma rays are able to travel through low Z
materials without any problems, however, high Z materials absorb and
deflect them. On the other hand, neutrons are able to pass easily
through high Z materials but are absorbed and scattered by low Z
materials.

\subsubsection{Gamma Ray Detection}
Gamma rays are produced from within the nuclei of radioactive
atoms. [3] To detect gamma rays, a scintillation detector is used. A
scintillation detector consists of a scintillator and photomultiplier
tube (PMT). A scintillator is a material which converts the energy
deposited by the gamma ray into optical photons (pulses of visible
light). The optical photons are picked up by the PMT which has two
functions: conversion of the optical photons into an electrical signal
and amplification of the signal so that it can be measured. The PMT
consists of a photocathode, dynodes and an anode. The photocathode
emits electrons when struck by the optical photons.  The electrons are
attracted to the first dynode which amplifies the electrons into more
electrons that are then accelerated to the second dynode which
produces even more electrons, and the process continues until the
electrons reach the anode. This is done so that a measurable current
pulse can be detected. The voltage of the current pulse is
proportional to the number of low energy photons which in turn is
proportional to the energy of the gamma ray. Each pulse height
produced from the PMT is analysed by a device called a Multi-Channel
Analyser (MCA). When the MCA receives a voltage pulse, it sorts it
into a bin depending on the energy and increases the counts in that
particular bin by one. A histogram representing the gamma ray spectrum
of an isotope is then drawn based on the information in the different
bins. [2][3]

(FIGURE 1: PICTURE OF A SCINTILLATION DETECTOR)

An alternative detector which can be used to measure gamma rays is a
semi-conductor detector. A voltage is placed between two terminals on
opposite sides of the semiconductor crystal. [3] When a gamma ray
strikes the crystal, its energy is given to the electron which is then
able to escape from the crystal. The electron is then attracted to the
positive electrode. When enough electrons are able to escape from the
crystal, a current will flow. The electric current has a voltage
proportional to the energy of the gamma ray. The voltage is then
sorted into a bin in the same way as the scintillation
detector. [2][3]

(FIGURE 2: PICTURE OF A NEUTRON DETECTOR)

\subsubsection{Neutron Detection}

Neutrons are uncharged particles. To detect neutrons, helium-3, an
isotope of helium is a good material to use. A tube consisting of
helium-3 gas linked to a power supply is a conventional neutron
detector. The tube contains charged plated or wires that are positive
and negative. When a low energy neutron strikes the detector, it is
absorbed by a helium-3 atom.  Energetic charged particles are then
produced which lose their energy by knocking off other helium-3
atoms. Positively charged particles are attracted to the negative
plate and the electrons are attracted to the positive plate. This
movement causes a small electric current to be generated which is then
recorded. Neutron detection cannot be used to identify isotopes
because they do not have characteristic lines that correspond to
discrete energies and they lose energy as they interact with a low
Z-material, thus their spectra is blurred. [3]

There are two techniques used for detection. These are passive and
active detection.The simplest detection method is passive detection
which involves detecting any and all radiation (gamma rays and
neutrons) that are spontaneously emitted by the warhead. [1] The
advantages of passive detectors are they require less electrical
output, they are not very expensive, they are easy to build and do not
add any additional health risks to workers. However, they are not
without their disadvantages. The major disadvantage is that the signal
emitted from a nuclear warhead needs to be greater than the background
signal in order for it to be recognised. As a result, there is a very
good chance that passive detection systems can easily be fooled if the
radioactive material is shielded to prevent measurable amounts of
radiation from reaching the detector [4]

The passive methods that are used to detect the presence of plutonium
depend on the emission of both neutrons and gamma rays from its
isotopes. The gamma rays emitted from the plutonium isotope are of
relatively low energy and attenuate greatly as they travel through the
warhead, thus gamma ray detection is not a reliable method to use. The
best technique to use is to look for neutrons emitted by the
plutonium-240 (Pu-240) isotope. The fissile isotope of plutonium is
plutonium-239 (Pu-239); however, Pu-240 is active also [5]. This
isotope spontaneously fissions and typically emits about 106 neutrons
per second with a characteristic energy of 1 MeV [5]. In spite of the
attenuation that these neutrons undergo as they travel through the
warhead, approximately 10\% emerge from the weapon and can be
detected. It has been estimated that a warhead containing 4kg of
Pu-239 contaminated by 6\% of Pu-240 could reliably be detected with a
detector 1m from the warhead in only 1 second [5]. The most important
advantage of using this method is that a well-defined image of the
warhead is not observed because the neutrons are scattered repeatedly
as they travel through the weapon to the outside, thus only the
grainiest geometrical information is revealed. [5][6]. A drawback of
using this method is that the emerging neutrons can be shielded
without any difficulty. These factors suggest that this neutron
detection technique is best used in examinations in which nuclear
material is expected to be present, for example, to verify that the
warheads earmarked to be dismantled are indeed nuclear.

The passive detection of uranium can be carried out by measuring the
high energy gamma rays that are emitted from the uranium-238 (U-238)
isotope. U-238 emits about 1MeV hard gamma rays at the rate of about
7.5 per gram per second with 1MeV energy [5]. Depleted uranium (almost
all U-238) also emits gamma rays at this rate but HEU emits gammas at
a rate of 0.5 gammas per gram-second with 1MeV energy. The amount of
depleted uranium and HEU determines the amount of gamma rays escaping
from the warhead. When a simple model of a nuclear warhead is used to
estimate this flux, the result is about 100 gamma rays per second
which is easily detectable [5]. The primary advantages of this
technique are that the counting rate (the number of gamma rays
detected in a particular time) is generally high. One disadvantage of
this technique is that the radiation signature of HEU is relatively
small; hence it quickly attenuates [7]. A second disadvantage is that
U-238 is used for other applications in which a high Z material is
essential, therefore there could be a lot of false positives (from
U-238 used in something that is not a nuclear weapon) if the method
was used at questionable sites.

\subsubsection{Existing Detection Equipment}
The existing detection technologies available are: 1. Radiation
Pagers: These are able to detect radiation at close distances. They
are lightweight and inexpensive, but are unable to ascertain the
source emitting the radiation.  [2] 2. Radioactive isotope
identification devices: These devices are able to identify
radioisotopes based on their gamma ray spectrum. These devices are
mainly handheld but heavy and delicate. They need to be cooled with
liquid nitrogen or by mechanical means, thus their usability in the
field is limited. They have a relatively short range for detecting
radiation sources with low radioactivity, notably shielded HEU, making
them unsuitable as the primary method of screening cargo
containers. [2] 3. Radiation Portal Monitors: Many of these devices
use large sheets of plastic scintillator material, such as polyvinyl
toluene (PVT), to detect radiation coming from a vehicle.  However,
PVT cannot identify the source of radiation. However, there are lots
of items in everyday trade that contain radioactive material. As a
result, some produce many false alarms, which may require considerable
effort to resolve, delaying the flow of commerce.  Newer versions have
some isotope identification capability. [2] They are primarily used
for security purposes at checkpoints and borders. [7]

\subsection{Active Detection}

\subsection{Muon Tomography}
\subsubsection{Cosmic Rays}
Cosmic rays are a flux of high energy particles that bombard the
earth’s atmosphere. They are produced in other parts of the universe
and approximately 98\% of these particles are protons or heavier
nuclei and 2\% are electrons. These cosmic rays collide with air
molecules and produce a shower of particles that include protons,
neutrons, electrons, positrons, photons, kaons and pions (both neutral
and charged). These particles interact by the nuclear and
electromagnetic forces to produce additional particles in a cascade
process. Pions will interact with air molecules via the strong force
but some will spontaneously decay via the weak force into a muon plus
a muon neutrino or an anti-muon neutrino. [1]
\begin{equation} \pi^+ \rightarrow \mu^+ + \overline{\nu}_\mu \end{equation}
\begin{equation} \pi^- \rightarrow \mu^- + \nu_\mu \end{equation}

\begin{figure}
  \includegraphics[width=\linewidth]{pile01/muon01.png}
  \caption{Cosmic ray cascade induced by a cosmic ray proton striking
    an air molecule nucleus. [1]}
\end{figure}

\subsubsection{Muons}
Muons are the most abundant charged particles at sea level. They are
produced high in the atmosphere, typically \unit{15}{\kilo\metre} and
lose about \unit{2}{\giga\electronvolt} before reaching the ground due
to ionisation. The mean energy of muons at the ground is
\unit{4}{\giga\electronvolt} [2]. They interact with matter via the
weak and electromagnetic forces but not with the strong force. They
decay via the weak force into an electron plus an electron neutrino or
an anti-electron neutrino.
\begin{equation} \mu^- \rightarrow e^- + \overline{\nu}_e + \nu_\mu \end{equation}
\begin{equation} \mu^+ \rightarrow e^+ + \nu_e + \overline{\nu}_\mu \end{equation}

The muon flux at sea level is about
\unit{1}{muon\usk\rpsquare{\centi\metre}\usk\reciprocal\minute} [3] or
\unit{10000}{muons\usk\rpsquare\metre\usk\reciprocal\minute}. They are
highly penetrating charged radiation.  A typical cosmic ray muon of
energy \unit{3}{\giga\electronvolt} can penetrate more than
\unit{1000}{\gram\usk\rpsquare{\centi\metre}} (e.g. 10 m of water). As
muons pass through matter they either scatter if they have high energy
or are absorbed if they have low energy. The angle at which they
scatter depends on the atomic number Z (number of protons) of the
material. As the atomic number of the material increases, the
scattering angle increases. In a layer \unit{10}{\centi\metre} thick,
a \unit{3}{\giga\electronvolt} muon will scatter with an angle of
\unit{2.3}{\milli\radian} in water, \unit{11}{\milli\radian} in iron
and \unit{20}{\milli\radian} in lead. [4]

\subsubsection{Limitations of X-rays}
X-ray radiography is successful in many areas but has
limitations. X-rays are unable to penetrate dense objects that have a
high atomic number. Multiple projections are needed in order to
resolve a three-dimensional structure using X-rays and they also pose
health risks from radiation. In X-ray radiography, absorption and
scattering cause attenuation of the incident beam which determines the
intensity of an image pixel. The maximum mean free path of photons is
about \unit{25}{\gram\usk\rpsquare{\centi\metre}} for all materials
which corresponds to \unit{2}{\centi\metre} of lead [4]. Even the most
penetrating gamma rays are attenuated by an e-folding in
\unit{2}{\centi\metre} of lead. A very large incident dose of
radiation is needed to penetrate thicker objects and that is harmful
for living organisms [5]. A different type of radiography must be used
for thicker objects and it must be based on the interaction of charged
particles with matter by multiple Coulomb scattering [4].

\subsubsection{Muon tomography concept}
Muon tomography is based on the multiple Coulomb scattering of muons
as they pass through a material. Radiographs of objects of any
thickness can be produced by using multiple scattering.  Cosmic ray
muons are passive and harmless radiation and allow radiograph of dense
objects with no artificial dose of radiation such as X-rays or gamma
rays. The scattering of muons differs significantly in three different
groups of materials: low Z (water, plastic, concrete), medium Z (iron,
copper) and high Z (lead, tungsten, uranium) [6]. Each muon carries
information about the objects it has penetrated and the properties of
these objects can be determined by measuring the scattering of
multiple muons. High Z objects can be detected amongst typical low Z
and medium Z objects. [3]

\begin{figure}
  \includegraphics[width=\linewidth]{pile01/muon02.png}
  \caption{Muon tomography concept. The grey tracks are the muons
    going through air and the black tracks are the muons that
    penetrate a dense object. [3]}
  \label{fig:muon02}
\end{figure}

The muon tomography concept is illustrated in
Figure~\ref{fig:muon02}. The position and angle of incoming muon
tracks are provided by a set of two or more planes of muon detectors
above and below the object. These detectors only detect vertically
oriented muons. Side detectors could be used to detect horizontally
oriented muons. The detectors above the object measure the position of
incident muons in two orthogonal coordinates. The scattering of the
muons that pass through the material depends on the type of the
object. The detectors below the object measure the positions and
angles of the scattered muons. The scattering angle of each muon is
calculated from the corresponding incident and scattered
measurements. The momentum is calculated from the slight scattering of
muons in the detectors themselves. [3]

\subsubsection{Simulations of muon tomography}
Simulations of muon tomography are very promising and results can be
obtained within a very short exposure time of approximately
\unit{1}{\minute}. The GEANT4 Monte Carlo package is used for the
simulations because it implements a complete, accurate and validated
model for multiple scattering. A detailed GEANT4 simulation of a
passenger van has been produced and reconstruction was achieved using
two different methods: mean and median. [3]

\begin{figure}
  \includegraphics[width=\linewidth]{pile01/muon03.png}
  \caption{Illustration of major objects in a simulated passenger van
    using GEANT4. The red block in the centre represents a
    \unit{10\times10\times10}{\cubic{\centi\metre}} solid piece of
    tungsten which is a high Z threat object. [3]}
\end{figure}
\begin{figure}
  \includegraphics[width=\linewidth]{pile01/muon04.png}
  \caption{Reconstruction of \unit{1}{\minute} of simulation muon
    exposure of the passenger van using the mean method. [3]}
  \label{fig:muon04}
\end{figure}
\begin{figure}
  \includegraphics[width=\linewidth]{pile01/muon05.png}
  \caption{Reconstruction of \unit{1}{\minute} of simulation muon
    exposure of the passenger van using the median method. [3]}
  \label{fig:muon05}
\end{figure}

The mean method of reconstruction shown in Figure~\ref{fig:muon04}
contains red spots scattered over the image.  The median method shown
in Figure~\ref{fig:muon05} does not contain these effects. The denser
components of the van (engine, battery, drive train) are shown as
green (low Z) or blue (medium) but the high Z threat object stands out
as red. The median method is clearly better. [3]

A ray crossing algorithm has been developed that highlights locations
where strongly scattered muons cross paths. The basis of the ray
crossing algorithm is the fact that a high Z object produces many
highly scattered rays which intersect in a small volume. A large depth
of medium Z material can also produce highly scattered rays but these
rays will spread over a larger volume. The algorithm was applied to a
simulated scene of a \unit{6\times2.4\times2.4}{\cubic\metre} cargo
container filled with \unit{12}{tons} of iron and three
\unit{9\times9\times12}{\cubic{\centi\metre}} uranium bricks were
buried within the iron. A cosmic ray exposure of \unit{1}{\minute} was
simulated and the tracks were processed using the ray crossing
algorithm. The results are shown in Figure~\ref{fig:muon06}. [6]

\begin{figure}
  \includegraphics[width=\linewidth]{pile01/muon06.png}
  \caption{Ray crossing algorithm reconstructions of \unit{1}{\minute}
    of simulated muon radiography of a
    \unit{6\times2.4\times2.4}{\cubic\metre} cargo container filled
    with \unit{12}{tons} of iron and three
    \unit{9\times9\times12}{\cubic{\centi\metre}} uranium bricks (a)
    and without the uranium bricks (b). [6]}
  \label{fig:muon06}
\end{figure}

All three uranium bricks are clearly identified in
Figure~\ref{fig:muon06}a. The image without the uranium bricks is empty of
any signal as shown in Figure~\ref{fig:muon06}b. The ray crossing
algorithm shows great promise in eliminating the scattering
background. [6]

Other simulations were also produced using a Monte Carlo simulation
code and the results are shown in Figure~\ref{fig:muon07}.

\begin{figure}
  \includegraphics[width=\linewidth]{pile01/muon07.png}
  \caption{Muon radiograph of a complex target in a volume of
    \unit{9\times3\times5.5}{\cubic\metre}. The first object (a) is a
    large complex lead sculpture. The reconstructed image (b) shows
    much of the detail of the object and it’s based on
    \unit{1}{\minute} of exposure. The second object (c) consists of a
    \unit{4\times2.4\times2.4}{\cubic\metre} container with walls of
    thickness equivalent to \unit{3}{\milli\metre} of steel. There are
    69 sheep made of water (shown in blue) inside the container with a
    body size of \unit{60\times30\times40}{\cubic{\centi\metre}} and
    three uranium bricks of size
    \unit{9\times9\times12}{\cubic{\centi\metre}} (shown in
    black). The reconstructed image (d), based on \unit{1}{\minute} of
    exposure, shows that the 3 uranium bricks stand out. The colour
    intensity in the two reconstructed images corresponds to the
    significance of the signal. [5]}
  \label{fig:muon07}
\end{figure}

\subsubsection{Experimental results of muon tomography}
There are a few prototype experimental muon tomography detectors that
show excellent results which are consistent with the simulations.  A
small scale experimental detector system was developed in 2003 at the
Los Alamos National Laboratory, Los Alamos, New Mexico [5]. A picture
of the detector is shown in Figure~\ref{fig:muon08}.

\begin{figure}
  \includegraphics[width=\linewidth]{pile01/muon08.png}
  \caption{Picture of experimental apparatus at the Los Alamos
    National Laboratory in 2003. There are four muon detectors
    labelled D1-D4 with a vertical spacing of
    \unit{27}{\centi\metre}. The detectors determine the positions and
    angles of the muons in two orthogonal coordinates (X and Y). The
    test object (W) was a tungsten cylinder of radius
    \unit{5.5}{\centi\metre} and height \unit{5.7}{\centi\metre}. A
    thick Lexan (L) plate of dimensions
    \unit{35\times60\times1}{\cubic{\centi\metre}} and steel support
    beams (B) were used to support the test object. [5]}
  \label{fig:muon08}
\end{figure}

Eight X and eight Y locations were measured for each muon by four
ionising radiation detectors contained in the detector stack. The two
detectors on top measure the incoming muon track while the two
detectors at the bottom measure the scattered track. Each delay line
drift chamber detector had an active area of
\unit{60\times60}{\square{\centi\metre}}. The detector was calibrated
with no test object to determine the precision of the position
measurement. A Windows based acquisition program was used to collect
the data. The reconstruction was approximated using the following
simple technique. Multiple scattered tracks were approximated to have
only a single scattering event and the point of scatter was located by
extrapolating the incident and scattered rays. A maximum likelihood
technique was used to assign voxels (3D pixels) to each scattered
muon. The reconstructed 3D image of the tungsten test object is shown
in Figure~\ref{fig:muon09}. [5]

\begin{figure}
  \includegraphics[width=\linewidth]{pile01/muon09.png}
  \caption{Test object reconstruction using 100\,000 muons. A
    volumetric image of \unit{1\times1\times1}{\cubic{\centi\metre}}
    voxels was reconstructed. The eight planes are horizontal slices
    near the middle of the volume, moving from top to bottom. Both the
    tungsten cylinder and the steel support beams are clearly
    visible. [5]}
  \label{fig:muon09}
\end{figure}

The data for the above image were collected over several hours because
the detector was not fully optimised. An optimised detector with 100\%
tracking efficiency and large solid angle could acquire the same data
in approximately \unit{30}{\minute}. The test object and the test
support beams can be clearly resolved using this long
run. Considerably shorter runs could be used for a simple yes/no
detection. [5]

Another sub-scale prototype was built at the Los Alamos National
Laboratory in 2006 called the Large Muon Tracker (LMT) which is 20'
tall. The design of this detector is very similar to the previous
detector. It consists of 6 top and 6 bottom planes of drift tube
detectors for each X and Y dimensions (24 planes in total) on a
flexible frame. The top and bottom sections are separated by
\unit{1.5}{\metre} to allow a large sampling region. X and Y tracks
are fitted separately to find the slope and intercept of each
dimension and combining them yields the 3D trajectory of the muon. A
picture of LMT is shown in Figure~\ref{fig:muon10}. [7]

\begin{figure}
  \includegraphics[width=\linewidth]{pile01/muon10.png}
  \caption{The Large Muon Tracker (LMT) prototype at the Los Alamos
    National Laboratory in 2006.  The precise positions of muon tracks
    above and below the sampling region are determined by the
    overlapping X and Y detector planes. The new redundant detector
    planes will be used improve the tracking efficiency and
    quality. [7]}
  \label{fig:muon10}
\end{figure}

The prototype of LMT was completed and tested in 2008. A simple
reconstruction technique was used to process the data. The sample
volume of \unit{1.5\times1.5\times1.0}{\cubic\metre} was segmented
into \unit{2\times2\times2}{\cubic{\centi\metre}} voxels. The median
scattering angle was calculated for all muons whose trajectories
intersected a voxel with an adjustable distance. The prototype was
tested using a \unit{10\times10\times10}{\cubic{\centi\metre}} lead
cube that represented the threat object and it was placed in the LMT
along with a car engine and transmission.  A photograph of the engine
in the LMT is shown in Figure~\ref{fig:muon11}. [8]

\begin{figure}
  \includegraphics[width=\linewidth]{pile01/muon11.png}
  \caption{Photograph of a car engine in the LMT at the Los Alamos
    National Laboratory in 2008. [8]}
  \label{fig:muon11}
\end{figure}

Data were collected for approximately \unit{160}{\minute} and have
been analysed to reconstruct the images shown in
Figure~\ref{fig:muon12}. The mean scattering angle is plotted for all
trajectories that pass through each voxel. [8]

\begin{figure}
  \includegraphics[width=\linewidth]{pile01/muon12.png}
  \caption{Mean scattering angle for a slice through the scene
    \unit{50}{\centi\metre} above the base plate. The left image shows
    the car engine, the middle image shows the engine with the lead
    cube and the right image shows the difference of the other two
    images. The lead block stands out dramatically. [8]}
  \label{fig:muon12}
\end{figure}

Another muon tomography prototype is located at the INFN National
Laboratories of Legnaro, Padova, Italy. A volume of
\unit{11}{\cubic\metre} can be inspected using the prototype which is
ideal for cargo inspection. A picture of the prototype is shown in
Figure~\ref{fig:muon13}. [9]

\begin{figure}
  \includegraphics[width=\linewidth]{pile01/muon13.png}
  \caption{Muon tomography system prototype located at the INFN
    National Laboratories of Legnaro. [9]}
  \label{fig:muon13}
\end{figure}

Two Muon Barrel drift chambers of dimensions
\unit{300\times250\times29}{\cubic{\centi\metre}}, built for the CMS
experiment at CERN, were used for the experiment, separated by
\unit{160}{\centi\metre}. A concrete and iron structure is supporting
the chambers. There are two additional drift chambers underneath the
bottom detector that will be used in the future as a momentum
filter. The reconstruction procedure uses a List Mode Iterative
Algorithm (LMIA) that process events one at a time instead of grouping
similar events together. [9]

\begin{figure}
  \includegraphics[width=\linewidth]{pile01/muon14.png}
  \caption{Test of the imaging capability of the prototype. The
    picture on the left shows the layout of iron bricks forming the
    word INFN and the picture on the right shows the result of the
    data analysis using the LMIA. The reconstructed image is very
    clear. [9]}
\end{figure}

The experiment was repeated using two lead blocks of dimensions
\unit{10\times10\times20}{\cubic{\centi\metre}} and two iron blocks of
dimensions \unit{10\times20\times20}{\cubic{\centi\metre}} placed on a
support structure \unit{65}{\centi\metre} in the vertical
direction. The 3D reconstruction of this layout is shown in
Figure~\ref{fig:muon15}.

\begin{figure}
  \includegraphics[width=\linewidth]{pile01/muon15.png}
  \caption{The left image is a sketch of the layout with the two lead
    and the two iron blocks. The darker blocks are the lead
    blocks. The right image shows the 3D view of the reconstructed
    image using the LMIA. [9]}
  \label{fig:muon15}
\end{figure}

The position of the blocks is reproduced correctly but there is finite
spatial resolution in the reconstruction especially in the vertical
direction. The reconstructed scattering density of the lead blocks is
greater than that of the iron blocks. It’s straightforward to
discriminate low Z or medium Z materials from high Z materials using
this method. The problem with this method is that discrimination
between high Z materials denser than iron is more difficult because of
the non-linearity in the reconstructed scattering density. This means
that the muon momentum has to be measured as well to allow a better
material recognition and increase the statistical precision of the
density measurement. [9]

\subsubsection{Applications of muon tomography}
Muon tomography could be used to protect the rail network from
terrorism. The idea is to equip train stations with large muon
detectors above and below. Density images can be produced very fast in
a time scale of minutes. High density objects such as nail bombs and
fissile materials will be easily identified. [10]

It could also be used as a detection method of nuclear devices or
material in vehicles and containers.  An automobile sized counting
station could be used to scan vehicles at border crossing. This would
allow examination of every vehicle and shipping container crossing a
border. It will require enough detectors to handle the traffic at the
borders. The total traffic crossing the US -- Mexico and the US --
Canada borders in 2008 was \(1.3\times10^8\) vehicles. Assuming a
single muon tomography detector could analyse a vehicle within
\unit{1}{\minute} and operates for 12 hours per day, then 500
detectors would be needed to handle the entire border crossing
traffic. This would cost a total of 1.5 to 2 billion dollars but its
negligible compared to the consequences of the detonation of a nuclear
bomb. A picture of how it could be implemented at a border crossing is
shown in Figure~\ref{fig:muon16}. [8]

\begin{figure}
  \includegraphics[width=\linewidth]{pile01/muon16.png}
  \caption{Schematic view of how a counting station might
    look. Vehicles would have to stop for approximately 20 seconds for
    the scan. [8]}
  \label{fig:muon16}
\end{figure}

Both methods could be used for nuclear dismantlement verification. The
vehicle transporting the bomb for disarmament could be scanned at
several stations during its journey to the dismantlement facilities. A
single muon tomography detector at the dismantlement facility could be
used to verify a small quantity of nuclear bombs. If there is a large
number of bombs queued for verification then the idea of the train
stations could be used. A room with muon detectors on the flood and
the ceiling could be used to scan all of them at the same time.

\citep{nuclearTamperSeals2001,equipmentIAEAinspectors2002,handbookSecBlunders2010}

\renewcommand{\refname}{\vspace*{-1em}\section{Bibliography}\vspace*{-1em}}
\bibliographystyle{plainnat}
\bibliography{references}

\end{document}
