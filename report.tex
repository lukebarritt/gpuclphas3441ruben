\documentclass[a4paper]{article}
\usepackage{mathtools}
\usepackage{natbib}
\usepackage{graphicx}
\usepackage{SIunits}

\begin{document}
\title{Nuclear Disarmament Verification}
\author{Aldridge, Barritt, Christodoulou, Lu, \\
	Moors, Rajroop, Theodorou, Xiao, \\
\small	Department of Physics and Astronomy, \\
\small	University College London, \\
\small	London WC1E 6BT, \\
\small	United Kingdom}
\date{2012 March 06\textsuperscript{th}}
\maketitle

\section*{Executive summary}
\subsection*{The Context}
  Why do we Verify?
We verify to assure confidence. We have a declaration, our verification methods, and our confidence. If we are too intrusive with our verification methods there is the threat of the spread of proliferation information which could in turn lead to a breakdown of the declaration agreements alongside an increased risk of mass devastation. If the verification methods employed are too feeble, the uncertainty in the confidence will be high and inspectors may fail to detect anomalous behaviour in the host’s disarmament activities.
This balance is known as the information barrier and it is spawn from the consideration of ratified treaties, the capabilities of the equipment and the feasibility of the procedure. The treaties outline what information should and shouldn’t be shared with the inspectors; the equipment and procedure are the technological or economic constraints.
*A chart of time, treaty and declaration information or disarmament rates, there should be an inflection at 2002 *
Summer’s treaty review
Nick’s information barrier in the context of treaty
Luke’s weapon introduction
Jack’s confidence in verification
\subsection*{The Process}
	What do we verify?
Luke’s weapon introduction
Nick’s brief overview 
Ralph’s dismantling process
Valentino’s chain of custody / Containment and surveillance of dismantlement 
Luke’s Blending down
\subsection*{The Methods}
	How do we verify?
Jenelle’s Passive detection; Gamma signature, Neutron signature
Kaijian’s detection methods; Induced Gamma and Neutron signature 
Nick’s Pit stuffing
Luke’s blending down
\subsection*{The Technology}
	What do we use to verify and how does it work?
Valentino’s Containment and surveillance / Ralph’s review of tags and seals tech
Kaijian’s detection methods
Jenelle’s Passive detection
Valentino’s Muon Tomography 
\subsection*{The Conclusion}
	(What are the Strengths weaknesses opportunities and threats?)
\subsection*{Final Summary}

\tableofcontents

\section{Introduction}
\subsection{Sample Section}
Here is a sample statement. \[ E = \gamma m_0 c^2 \]

\section{Political Background}
\subsection{Treaties}

\section{Detection Schemes}
\subsection{Passive Detection}
\subsection{Active Detection}

\subsection{Muon Tomography}
\subsubsection{Cosmic Rays}
Cosmic rays are a flux of high energy particles that bombard the earth’s atmosphere. They are
produced in other parts of the universe and approximately 98\% of these particles are protons or
heavier nuclei and 2\% are electrons. These cosmic rays collide with air molecules and produce a
shower of particles that include protons, neutrons, electrons, positrons, photons, kaons and pions
(both neutral and charged). These particles interact by the nuclear and electromagnetic forces to
produce additional particles in a cascade process. Pions will interact with air molecules via the
strong force but some will spontaneously decay via the weak force into a muon plus a muon
neutrino or an anti-muon neutrino. [1]
\begin{equation} \pi^+ \rightarrow \mu^+ + \overline{\nu}_\mu \end{equation}
\begin{equation} \pi^- \rightarrow \mu^- + \nu_\mu \end{equation}

\begin{figure}[h!]
  \includegraphics{pile01/muon01.png}
  \caption{Cosmic ray cascade induced by a cosmic ray proton striking
an air molecule nucleus. [1]}
\end{figure}

\subsubsection{Muons}
Muons are the most abundant charged particles at sea level. They are produced high in the
atmosphere, typically \unit{15}{\kilo\metre} and lose about \unit{2}{\giga\electronvolt} before reaching the ground due to ionization. The
mean energy of muons at the ground is \unit{4}{\giga\electronvolt} [2]. They interact with matter via the weak and
electromagnetic forces but not with the strong force. They decay via the weak force into an electron
plus an electron neutrino or an anti-electron neutrino.
\begin{equation} \mu^- \rightarrow e^- + \overline{\nu}_e + \nu_\mu \end{equation}
\begin{equation} \mu^+ \rightarrow e^+ + \nu_e + \overline{\nu}_\mu \end{equation}

The muon flux at sea level is about \unit{1}{muon\usk\rpsquare{\centi\metre}\usk\reciprocal\minute} [3] or 
\unit{10000}{muons\usk\rpsquare\metre\usk\reciprocal\minute}. They are highly penetrating charged radiation. 
A typical cosmic ray muon of energy \unit{3}{\giga\electronvolt} can penetrate more than 
\unit{1000}{\gram\usk\rpsquare{\centi\metre}} (eg. 10 m of water). As muons pass through matter they either
scatter if they have high energy or are absorbed if they have low energy. The angle at which they
scatter depends on the atomic number Z (number of protons) of the material. As the atomic number
of the material increases, the scattering angle increases. In a layer \unit{10}{\centi\metre} thick, a 
\unit{3}{\giga\electronvolt} muon will scatter with an angle of \unit{2.3}{\milli\radian} in water, 
\unit{11}{\milli\radian} in iron and \unit{20}{\milli\radian} in lead. [4]

\subsubsection{Limitations of X-rays}
X-ray radiography is successful in many areas but has limitations. X-rays are unable to penetrate
dense objects that have a high atomic number. Multiple projections are needed in order to resolve a
three-dimensional structure using X-rays and they also pose health risks from radiation. In X-ray
radiography, absorption and scattering cause attenuation of the incident beam which determines
the intensity of an image pixel. The maximum mean free path of photons is about \unit{25}{\gram\usk\rpsquare{\centi\metre}} for all
materials which corresponds to \unit{2}{\centi\metre} of lead [4]. Even the most penetrating gamma rays are
attenuated by an e-folding in \unit{2}{\centi\metre} of lead. A very large incident dose of radiation is needed to
penetrate thicker objects and that is harmful for living organisms [5]. A different type of radiography
must be used for thicker objects and it must be based on the interaction of charged particles with
matter by multiple Coulomb scattering [4].

\subsubsection{Muon tomography concept}
Muon tomography is based on the multiple Coulomb scattering of muons as they pass through a
material. Radiographs of objects of any thickness can be produced by using multiple scattering.
Cosmic ray muons are passive and harmless radiation and allow radiograph of dense objects with no
artificial dose of radiation such as X-rays or gamma rays. The scattering of muons differs significantly
in three different groups of materials: low Z (water, plastic, concrete), medium Z (iron, copper) and
high Z (lead, tungsten, uranium) [6]. Each muon carries information about the objects it has
penetrated and the properties of these objects can be determined by measuring the scattering of
multiple muons. High Z objects can be detected amongst typical low Z and medium Z objects. [3]

\begin{figure}[h!]
  \includegraphics{pile01/muon02.png}
  \caption{Muon tomography concept. The grey tracks are the muons going through air and the black
tracks are the muons that penetrate a dense object. [3]}
  \label{muon02}
\end{figure}

The muon tomography concept is illustrated above in Figure~\ref{muon02}. The position and angle of incoming
muon tracks are provided by a set of two or more planes of muon detectors above and below the
object. These detectors only detect vertically oriented muons. Side detectors could be used to
detect horizontally oriented muons. The detectors above the object measure the position of incident
muons in two orthogonal coordinates. The scattering of the muons that pass through the material
depends on the type of the object. The detectors below the object measure the positions and angles
of the scattered muons. The scattering angle of each muon is calculated from the corresponding
incident and scattered measurements. The momentum is calculated from the slight scattering of
muons in the detectors themselves. [3]

\subsubsection{Simulations of muon tomography}
Simulations of muon tomography are very promising and results can be obtained within a very short
exposure time of approximately \unit{1}{\minute}. The GEANT4 Monte Carlo package is used for the simulations
because it implements a complete, accurate and validated model for multiple scattering. A detailed
GEANT4 simulation of a passenger van has been produced and reconstruction was achieved using
two different methods: mean and median. [3]

\begin{figure}[h!]
  \includegraphics{pile01/muon03.png}
  \caption{Illustration of major objects in a simulated passenger van using GEANT4. The red block in
the centre represents a \unit{10\times10\times10}{\cubic{\centi\metre}} solid piece of tungsten which is a high Z threat object. [3]}
\end{figure}
\begin{figure}[h!]
  \includegraphics{pile01/muon04.png}
  \caption{Reconstruction of \unit{1}{\minute} of simulation muon exposure of the passenger van using the
mean method. [3]}
  \label{muon04}
\end{figure}
\begin{figure}[h!]
  \includegraphics{pile01/muon05.png}
  \caption{Reconstruction of \unit{1}{\minute} of simulation muon exposure of the passenger van using the
median method. [3]}
  \label{muon05}
\end{figure}

The mean method of reconstruction shown on Figure~\ref{muon04} contains red spots scattered over the image.
The median method shown on Figure~\ref{muon05} does not contain these effects. The denser components of
the van (engine, battery, drive train) are shown as green (low Z) or blue (medium) but the high Z
threat object stands out as red. The median method is clearly better. [3]

A ray crossing algorithm has been developed that highlights locations where strongly scattered
muons cross paths. The basis of the ray crossing algorithm is the fact that a high Z object produces
many highly scattered rays which intersect in a small volume. A large depth of medium Z material
can also produce highly scattered rays but these rays will spread over a larger volume. The algorithm
was applied to a simulated scene of a \unit{6\times2.4\times2.4}{\cubic\metre} cargo container filled with \unit{12}{tons} of iron and
three \unit{9\times9\times12}{\cubic{\centi\metre}} uranium bricks were buried within the iron. A cosmic ray exposure of \unit{1}{\minute} was
simulated and the tracks were processed using the ray crossing algorithm. The results are shown
below on Figure~\ref{muon06}. [6]

\begin{figure}[h!]
  \includegraphics{pile01/muon06.png}
  \caption{Ray crossing algorithm reconstructions of \unit{1}{\minute} of simulated muon radiography of a
\unit{6\times2.4\times2.4}{\cubic\metre} cargo container filled with \unit{12}{tons} of iron and
three \unit{9\times9\times12}{\cubic{\centi\metre}} uranium bricks (a) and without the uranium bricks (b). [6]}
  \label{muon06}
\end{figure}

All three uranium bricks are clearly identified on Figure~\ref{muon06}a. The image without the uranium bricks is
empty of any signal as shown on Figure~\ref{muon06}b. The ray crossing algorithm shows great promise in
eliminating the scattering background. [6]

Other simulations were also produced using a Monte Carlo simulation code and the results are
shown below on Figure~\ref{muon07}.

\begin{figure}[h!]
  \includegraphics{pile01/muon07.png}
  \caption{Muon radiograph of a complex target in a volume of \unit{9\times3\times5.5}{\cubic\metre}. The first object (a) is a
large complex lead sculpture. The reconstructed image (b) shows much of the detail of the object
and it’s based on \unit{1}{\minute} of exposure. The second object (c) consists of a \unit{4\times2.4\times2.4}{\cubic\metre} container with
walls of thickness equivalent to \unit{3}{\milli\metre} of steel. There are 69 sheep made of water (shown in blue)
inside the container with a body size of \unit{60\times30\times40}{\cubic{\centi\metre}} and three uranium bricks of 
size \unit{9\times9\times12}{\cubic{\centi\metre}} (shown in black). The reconstructed image (d), based 
on \unit{1}{\minute} of exposure, shows that the 3 uranium bricks stand out. The colour intensity in the two reconstructed 
images corresponds to the significance of the signal. [5]}
  \label{muon07}
\end{figure}

\subsubsection{Experimental results of muon tomography}
There are a few prototype experimental muon tomography detectors that show excellent results
which are consistent with the simulations.
A small scale experimental detector system was developed in 2003 at the Los Alamos National
Laboratory, Los Alamos, New Mexico [5]. A picture of the detector is shown below Figure~\ref{muon08}.

\begin{figure}[h!]
  \includegraphics{pile01/muon08.png}
  \caption{Picture of experimental apparatus at the Los Alamos National Laboratory in 2003. There
are four muon detectors labelled D1-D4 with a vertical spacing of \unit{27}{\centi\metre}. The detectors determine
the positions and angles of the muons in two orthogonal coordinates (X and Y). The test object (W)
was a tungsten cylinder of radius \unit{5.5}{\centi\metre} and height \unit{5.7}{\centi\metre}. A thick Lexan (L) plate of dimensions
\unit{35\times60\times1}{\cubic{\centi\metre}} and steel support beams (B) were used to support the test object. [5]}
  \label{muon08}
\end{figure}

Eight X and eight Y locations were measured for each muon by four ionizing radiation detectors
contained in the detector stack. The two detectors on top measure the incoming muon track while
the two detectors at the bottom measure the scattered track. Each delay line drift chamber detector
had an active area of \unit{60\times60}{\square{\centi\metre}}. The detector was calibrated with no test object to determine the
precision of the position measurement. A Windows based acquisition program was used to collect
the data. The reconstruction was approximated using the following simple technique. Multiple
scattered tracks were approximated to have only a single scattering event and the point of scatter
was located by extrapolating the incident and scattered rays. A maximum likelihood technique was
used to assign voxels (3D pixels) to each scattered muon. The reconstructed 3D image of the
tungsten test object is shown below on Figure~\ref{muon09}. [5]

\begin{figure}[h!]
  \includegraphics{pile01/muon09.png}
  \caption{Test object reconstruction using 100\,000 muons. A volumetric image of \unit{1\times1\times1}{\cubic{\centi\metre}} voxels
was reconstructed. The eight planes are horizontal slices near the middle of the volume, moving
from top to bottom. Both the tungsten cylinder and the steel support beams are clearly visible. [5]}
  \label{muon09}
\end{figure}

The data for the above image were collected over several hours because the detector was not fully
optimised. An optimised detector with 100\% tracking efficiency and large solid angle could acquire
the same data in approximately \unit{30}{\minute}. The test object and the test support beams can be clearly
resolved using this long run. Considerably shorter runs could be used for a simple yes/no detection. [5]

Another sub-scale prototype was built at the Los Alamos National Laboratory in 2006 called the
Large Muon Tracker (LMT) which is 20’ tall. The design of this detector is very similar to the previous
detector. It consists of 6 top and 6 bottom planes of drift tube detectors for each X and Y dimensions
(24 planes in total) on a flexible frame. The top and bottom sections are separated by \unit{1.5}{\metre} to allow
a large sampling region. X and Y tracks are fitted separately to find the slope and intercept of each
dimension and combining them yields the 3D trajectory of the muon. A picture of LMT is shown
below on Figure~\ref{muon10}. [7]

\begin{figure}[h!]
  \includegraphics{pile01/muon10.png}
  \caption{The Large Muon Tracker (LMT) prototype at the Los Alamos National Laboratory in 2006.
The precise positions of muon tracks above and below the sampling region are determined by the
overlapping X and Y detector planes. The new redundant detector planes will be used improve the
tracking efficiency and quality. [7]}
  \label{muon10}
\end{figure}

The prototype of LMT was completed and tested in 2008. A simple reconstruction technique was
used to process the data. The sample volume of \unit{1.5\times1.5\times1.0}{\cubic\metre} was segmented into \unit{2\times2\times2}{\cubic{\centi\metre}}
voxels. The median scattering angle was calculated for all muons whose trajectories intersected a
voxel with an adjustable distance. The prototype was tested using a \unit{10\times10\times10}{\cubic{\centi\metre}} lead cube that
represented the threat object and it was placed in the LMT along with a car engine and transmission.
A photograph of the engine in the LMT is shown below on Figure~\ref{muon11}. [8]

\begin{figure}[h!]
  \includegraphics{pile01/muon11.png}
  \caption{Photograph of a car engine in the LMT at the Los Alamos National Laboratory in 2008. [8]}
  \label{muon11}
\end{figure}

Data were collected for approximately \unit{160}{\minute} and have been analysed to reconstruct the images
shown below on Figure~\ref{muon12}. The mean scattering angle is plotted for all trajectories that pass through
each voxel. [8]

\begin{figure}[h!]
  \includegraphics{pile01/muon12.png}
  \caption{Mean scattering angle for a slice through the scene \unit{50}{\centi\metre} above the base plate. The left
image shows the car engine, the middle image shows the engine with the lead cube and the right
image shows the difference of the other two images. The lead block stands out dramatically. [8]}
  \label{muon12}
\end{figure}

Another muon tomography prototype is located at the INFN National Laboratories of Legnaro,
Padova, Italy. A volume of \unit{11}{\cubic\metre} can be inspected using the prototype which is ideal for cargo
inspection. A picture of the prototype is shown below on Figure~\ref{muon13}. [9]

\begin{figure}[h!]
  \includegraphics{pile01/muon13.png}
  \caption{Muon tomography system prototype located at the INFN National Laboratories of
Legnaro. [9]}
  \label{muon13}
\end{figure}

Two Muon Barrel drift chambers of dimensions \unit{300\times250\times29}{\cubic{\centi\metre}}, built for the CMS experiment at
CERN, were used for the experiment, separated by \unit{160}{\centi\metre}. A concrete and iron structure is
supporting the chambers. There are two additional drift chambers underneath the bottom detector
that will be used in the future as a momentum filter. The reconstruction procedure uses a List Mode
Iterative Algorithm (LMIA) that process events one at a time instead of grouping similar events
together. [9]

\begin{figure}[h!]
  \includegraphics{pile01/muon14.png}
  \caption{Test of the imaging capability of the prototype. The picture on the left shows the layout of
iron bricks forming the word INFN and the picture on the right shows the result of the data analysis
using the LMIA. The reconstructed image is very clear. [9]}
  \label{muon14}
\end{figure}

The experiment was repeated using two lead blocks of dimensions \unit{10\times10\times20}{\cubic{\centi\metre}} and two iron
blocks of dimensions \unit{10\times20\times20}{\cubic{\centi\metre}} placed on a support structure \unit{65}{\centi\metre} in the vertical direction. The
3D reconstruction of this layout is shown below on Figure~\ref{muon15}.

\begin{figure}[h!]
  \includegraphics{pile01/muon15.png}
  \caption{The left image is a sketch of the layout with the two lead and the two iron blocks. The
darker blocks are the lead blocks. The right image shows the 3D view of the reconstructed image
using the LMIA. [9]}
  \label{muon15}
\end{figure}

The position of the blocks is reproduced correctly but there is finite spatial resolution in the
reconstruction especially in the vertical direction. The reconstructed scattering density of the lead
blocks is greater than that of the iron blocks. It’s straightforward to discriminate low Z or medium Z
materials from high Z materials using this method. The problem with this method is that
discrimination between high Z materials denser than iron is more difficult because of the non-
linearity in the reconstructed scattering density. This means that the muon momentum has to be
measured as well to allow a better material recognition and increase the statistical precision of the
density measurement. [9]

\subsubsection{Applications of muon tomography}
Muon tomography could be used to protect the rail network from terrorism. The idea is to equip
train stations with large muon detectors above and below. Density images can be produced very fast
in a time scale of minutes. High density objects such as nail bombs and fissile materials will be easily
identified. [10]

It could also be used as a detection method of nuclear devices or material in vehicles and containers.
An automobile sized counting station could be used to scan vehicles at border crossing. This would
allow examination of every vehicle and shipping container crossing a border. It will require enough
detectors to handle the traffic at the borders. The total traffic crossing the US -- Mexico and the US --
Canada borders in 2008 was \(1.3\times10^8\) vehicles. Assuming a single muon tomography detector could
analyse a vehicle within \unit{1}{\minute} and operates for 12 hours per day, then 500 detectors would be
needed to handle the entire border crossing traffic. This would cost a total of 1.5 to 2 billion dollars
but its negligible compared to the consequences of the detonation of a nuclear bomb. A picture of
how it could be implemented at a border crossing is shown below on Figure~\ref{muon16}. [8]

\begin{figure}[h!]
  \includegraphics{pile01/muon16.png}
  \caption{Schematic view of how a counting station might look. Vehicles would have to stop for
approximately 20 seconds for the scan. [8]}
  \label{muon16}
\end{figure}

Both methods could be used for nuclear dismantlement verification. The vehicle transporting the
bomb for disarmament could be scanned at several stations during its journey to the dismantlement
facilities. A single muon tomography detector at the dismantlement facility could be used to verify a
small quantity of nuclear bombs. If there is a large number of bombs queued for verification then
the idea of the train stations could be used. A room with muon detectors on the flood and the ceiling
could be used to scan all of them at the same time.

\end{document}
